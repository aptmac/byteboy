% TEMPLATE for Usenix papers, specifically to meet requirements of
%  USENIX '05
% originally a template for producing IEEE-format articles using LaTeX.
%   written by Matthew Ward, CS Department, Worcester Polytechnic Institute.
% adapted by David Beazley for his excellent SWIG paper in Proceedings,
%   Tcl 96
% turned into a smartass generic template by De Clarke, with thanks to
%   both the above pioneers
% use at your own risk.  Complaints to /dev/null.
% make it two column with no page numbering, default is 10 point

% Munged by Fred Douglis <douglis@research.att.com> 10/97 to separate
% the .sty file from the LaTeX source template, so that people can
% more easily include the .sty file into an existing document.  Also
% changed to more closely follow the style guidelines as represented
% by the Word sample file. 

% Note that since 2010, USENIX does not require endnotes. If you want
% foot of page notes, don't include the endnotes package in the 
% usepackage command, below.

% This version uses the latex2e styles, not the very ancient 2.09 stuff.
\documentclass[letterpaper,twocolumn,10pt]{article}
\usepackage{usenix,epsfig,endnotes}
\begin{document}

%don't want date printed
\date{}

%make title bold and 14 pt font (Latex default is non-bold, 16 pt)
\title{\Large \bf Byteboy : A Byteman Companion Application}

\author{
{\rm Alex P. T. Macdonald}\\
apmacdon@sfu.ca\\
Simon Fraser University
\and
{\rm Ian Pun}\\
Simon Fraser University
} % end author

\maketitle

% Use the following at camera-ready time to suppress page numbers.
% Comment it out when you first submit the paper for review.
\thispagestyle{empty}


\subsection*{Abstract}
Your Abstract Text Goes Here.  Just a few facts.
Whet our appetites.

\section{Introduction}

A paragraph of text goes here.  Lots of text.  Plenty of interesting
text. \\

More fascinating text. Wow this paper is so interesting. It's incredible.\\

\section{Background}

Some embedded literal typset code might 
look like the following :

{\tt \small
\begin{verbatim}
int wrap_fact(ClientData clientData,
              Tcl_Interp *interp,
              int argc, char *argv[]) {
    int result;
    int arg0;
    if (argc != 2) {
        interp->result = "wrong # args";
        return TCL_ERROR;
    }
    arg0 = atoi(argv[1]);
    result = fact(arg0);
    sprintf(interp->result,"%d",result);
    return TCL_OK;
}
\end{verbatim}
}

% This will be useful for citations, don't remove just yet.
Now we're going to cite somebody.  Watch for the cite tag.
Here it comes~\cite{Chaum1981,Diffie1976}.  The tilde character (\~{})
in the source means a non-breaking space.  This way, your reference will
always be attached to the word that preceded it, instead of going to the
next line.

\section{Motivation}
\subsection{A Poem}

Roses are Red. \\
Violets are Blue. \\
Omae wa mou shindeiru \\

% HELPFUL STUFF ON HOW TO EMBED IMAGES
% you can also use the wonderful epsfig package...
% \begin{figure}[t]
% \begin{center}
% \begin{picture}(300,150)(0,200)
% \put(-15,-30){\special{psfile = fig1.ps hscale = 50 vscale = 50}}
% \end{picture}\\
% \end{center}
% \caption{Wonderful Flowchart}
% \end{figure}

% This text came after the figure, so we'll casually refer to Figure 1
% as we go on our merry way.

\subsection{Here's some neat stuff I'm not going to remove yet because it'll be pretty useful for our motivating example I think when showing how to run the code}

Sometimes you want to really call attention to a piece of text.  You
can center it in the column like this:
\begin{center}
{\tt \_1008e614\_Vector\_p}
\end{center}
and people will really notice it.\\

\noindent
The noindent at the start of this paragraph makes it clear that it's
a continuation of the preceding text, not a new para in its own right.


Now this is an ingenious way to get a forced space.
{\tt Real~$*$} and {\tt double~$*$} are equivalent. 

Now here is another way to call attention to a line of code, but instead
of centering it, we noindent and bold it.\\

\noindent
{\bf \tt size\_t : fread ptr size nobj stream } \\

And here we have made an indented para like a definition tag (dt)
in HTML.  You don't need a surrounding list macro pair.
\begin{itemize}
\item[]  {\tt fread} reads from {\tt stream} into the array {\tt ptr} at
most {\tt nobj} objects of size {\tt size}.   {\tt fread} returns
the number of objects read. 
\end{itemize}
This concludes the definitions tag.


\section{Approach}

This is where our approach will go. \\

\section{Evaluation \& Observations}

This is where our evaluation \& observations will go. \\

\section{Related Work}

This is where related work goes. \\

\section{Conclusion \& Future Work}

This is where the conclusion goes. \\

\section{Acknowledgments}

Good night Springton, there will be no encores. \\

\begin{thebibliography}{99}

\bibitem{bytemanwebsite} 
Byteman: Simplify Java tracing, monitoring and testing with Byteman,
\\\texttt{http://byteman.jboss.org/}

\bibitem{bytemanguide}
Byteman Programmer's Guide, 3.0.10, Apr 27, 2017
\\\texttt{http://downloads.jboss.org/byteman/3.0.10/ProgrammersGuide.html}

\bibitem{yo}
Yo Code - Extension Generator
\\\texttt{https://code.visualstudio.com/docs/extensions/yocode}

\bibitem{textmate}
Textmate: Language Grammars
\\\texttt{https://manual.macromates.com/en/language\_grammars}

\bibitem{bytemanvideo}
Dinn, Andrew. \\\textit{Monitoring Application-Specific Behavior Using Thermostat and Byteman}
YouTube. 2016. URL: \\\texttt{https://www.youtube.com/watch?v=teL7qnulUTM}

\end{thebibliography}

\end{document}






